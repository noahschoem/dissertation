\section{Forcing}
\label{forcing:forcing}

We will assume knowledge of forcing; for a detailed summary, see for instance \cite{jech_set_theory}, Chapter 13.
Here we just recall the forcing and iterated forcing specifics necessary for our purposes.

Just to review, here is the core theorem of forcing:

\begin{theorem}[Forcing and Generic Model Theorem]
  Let $V$ be a universe of $\mathrm{ZFC}$ and let $(\mathbb{P},<)$ be a notion of forcing.
  Then there is a universe $V^\mathbb{P}=V[G]$ of $\mathrm{ZFC}$ given by the addition of a new set $G$ such that
  \begin{enumerate}
    \item $Ord^V=Ord^{V[G]}$
    \item If $W\supseteq V$, $W\models ZFC$ and $G\in W$, then $V[G]\subseteq W$
    \item $G$ is $\mathbb{P}$-generic over $V$, that is, for every dense subset (equivalently, every maximal antichain) $D$ of $\mathbb{P}$ in $V$, $G\cap D\neq\emptyset$
  \end{enumerate}
  And the theory of $V[G]$ may be understood by the forcing relation $\forces_{\mathbb{P}}$, entirely definable in $V$, because for each formula $\varphi$,
  \[
    V[G]\models \varphi \iff \exists p\in G \ p\forces \varphi
  \]
\end{theorem}

\begin{remark}
  Here are some notations and conventions:
  \begin{itemize}
    \item We adopt the convention that for $p$, $q\in \mathbb{P}$, $q\leq p$ means that $q$ is a \emph{stronger} condition than $p$.
    
    \item If $p$, $q$ are compatible, we write $p\compat q$ to denote this, otherwise $p$, $q$ are incompatible, denoted by $p\incompat q$.
    
    \item If $p\compat q$, we use $p\land q$ to denote the weakest $r$ such that $r\leq p$ and $r\leq q$.
    
    \item For any $p$, $q$, we use $p\lor q$ to denote the strongest $r$ such that $p\leq r$ and $q\leq r$.
  \end{itemize}
\end{remark}

If $\tau$ is an ordinal and $\tau\in Reg^V$ (i.e. $\tau$ is a regular cardinal in $V$), it is not necessarily the case that $\tau\in Reg^{V^{\mathbb{P}}}$ (or is even a cardinal).
Certain nice combinatorial properties guarantee that $\tau$ remains a cardinal after forcing.
We outline some key such properties:

\begin{definition}[Chain condition, presaturation, and closure]
  Let $(\mathbb{P},\leq)$ be a poset and let $\tau$ be a cardinal. We say that:
  \begin{enumerate}[label={(\roman*)}]
    \item $\mathbb{P}$ has the \emph{$\tau$-chain condition} (is $\tau$-cc) if every antichain of $\mathbb{P}$ is of size less than $\tau$
    
    \item $\mathbb{P}$ is \emph{$\tau$-directed closed} (i.e $<\tau$-directed closed, written $\tau$-dc, $<\tau$-dc) if whenever $D\subseteq \mathbb{P}$ is a directed set\footnote{that is, for all $p,q\in D$, there is an $r\in D$ such that $r\leq p,q$} with $|D|<\tau$, there is a $q\in \mathbb{P}$ such that whenever $p\in D$, $q\leq p$
    
    \item $\mathbb{P}$ is \emph{$\tau$-closed} (i.e. $<\tau$-closed) if whenever $\rho<\tau$ and $\seq{p_\alpha \mid \alpha<\rho}$ is a $\leq$-decreasing sequence in $\mathbb{P}$, there is a $p\in\mathbb{P}$ such that $p\leq p_\alpha$ for all $\alpha<\rho$
    
    \item $\mathbb{P}$ is \emph{$\tau$-distributive} if for every $\dot{f}:\check{\tau}\to\check{V}$ in $V^{\mathbb{P}}$, $\dot{f}\in \check{V}$; equivalently if the intersection of $\tau$-many open dense subsets of $\mathbb{P}$ is dense\footnote{
      Note this is unlike the other properties here in that this is an $=\tau$ principle as opposed to a $<\tau$ principle.
      We write ``$<\tau$-distributive" to mean ``for all $\tau'<\tau$, $\mathbb{P}$ is $\tau'$-distributive".
    }
    
    \item $\mathbb{P}$ is \emph{$\tau$-presaturated} (i.e. $<\tau$-presaturated) if for every $\lambda<\tau$ and every family $\seq{A_\alpha \mid \alpha<\lambda}$ of antichains,
    there are densely many $p\in\mathbb{P}$ such that for all $\alpha$, $\{q\in A_\alpha \mid p\compat q\}$ has cardinality $<\tau$. Note that $\tau$-cc implies $\tau$-presaturation.    

    \item $\mathbb{P}$ is \emph{$\tau$-preserving} if $V^\mathbb{P}\models ``\check{\tau}\text{ is a cardinal}"$
  \end{enumerate}
\end{definition}

\begin{proposition}
  If $\mathbb{P}$ is $\tau$-cc, $\tau$-dc, $\tau$-closed, $<\tau$-distributive, $\tau$-distributive, or $\tau$-presaturated, then $\mathbb{P}$ is $\tau$-preserving.
\end{proposition}

The proofs are standard; see \cite{jech_set_theory}, Chapters 14 and 15, for more details.

The usual posets for collapsing cardinals are the L\'evy collapsing posets $Col(\tau,\lambda)$ and $Col(\tau,<\lambda)$.
In brief, $Col(\tau,\lambda)$ is a $\tau$-directed closed, $\lambda^+$-cc forcing notion such that for every $\rho\in Card^V \cap [\tau,\lambda]$, $|\rho]^{V^{Col(\tau,\lambda)}}=\tau$
If $\lambda$ is inaccessible, then $Col(\tau,<\lambda)$ is a $\tau$-directed closed, $\lambda$-cc forcing such that for every $\rho\in Card^V \cap [\tau,\lambda)$, $|\rho]^{V^{Col(\tau,\lambda)}}=\tau$.

Often, we will wish to force with multiple forcings simultaneously, or in succession.
If both $\mathbb{P}$ and $\mathbb{Q}$ are in $V$, then we will want to force with their \emph{product}; if instead $\mathbb{Q}\in V^{\mathbb{P}}$, then we will use \emph{iteration}.

\cite{jech_set_theory}, Chapter 15, is a standard reference for product forcing.
For iteration forcing, this section draws from \cite{baumgartner_iterated_forcing}.

\begin{definition}[Product Forcing]
  If $\mathbb{P}$ and $\mathbb{Q}$ are in $V$, then the poset $\mathbb{P}\times \mathbb{Q}$ consists of conditions of the form $(p,q)$ where $p\in \mathbb{P}$ and $q\in \mathbb{Q}$, and $(p',q')\leq (p,q)$ if $p'\leq p$ and $q'\leq q$.
\end{definition}

The following forcing theorem characterizes what a $\mathbb{P}\times \mathbb{Q}$-generic looks like:

\begin{theorem}[Product Theorem]
  Let $G\subseteq \mathbb{P}\times \mathbb{Q}$.
  Then $G$ is $\mathbb{P}\times \mathbb{Q}$-generic over $V$ if and only if $G=G_1 \times G_2$ where $G_1$ is $\mathbb{P}$-generic over $V$ and $G_2$ is $\mathbb{Q}$-generic over $V^{\mathbb{P}}$, and in such case $V[G]=V[G_1][G_2]$.
\end{theorem}

Larger products are possible, but beyond our scope.
As for properties preserved by product forcing:
\begin{proposition}
  Let $\tau$ be a regular cardinal, and suppose $\mathbb{P}$ and $\mathbb{Q}$ are both $\tau$-closed. Then $\mathbb{P}\times \mathbb{Q}$ is also $\tau$-closed.
\end{proposition}

\begin{proposition}[Easton's Lemma, c.f. \cite{jech_set_theory}, Lemma 15.19]\thlabel{product_cc_closed}
  Let $\tau$ be regular, and suppose $\mathbb{P}$ is $\tau^+$-cc and $\mathbb{Q}$ is $\tau^+$-closed.
  Then in $V^{\mathbb{P}}$, $\mathbb{Q}$ remains $\tau$-distributive.
\end{proposition}

In the event that $\mathbb{Q}\in V^{\mathbb{P}}\setminus V$, we may use the notion of two-step iteration:

\begin{definition}
  The two-step iteration $\mathbb{P} * \dot{\mathbb{Q}}$ is given by conditions of the form $(p,\dot{q})$ where $p\in\mathbb{P}$ and $p\forces_{\mathbb{P}}\dot{q}\in \dot{\mathbb{Q}}$, and $(p',\dot{q}')\leq (p,q)$ if $p'\leq p$ and $p'\forces_{\mathbb{P}} \dot{q}'\leq \dot{q}$
\end{definition}

The following characterizes generic extensions of two-step iterations:
\begin{proposition}
  Let $G\subseteq \mathbb{P} * \dot{\mathbb{Q}}$.
  Then $G$ is $\mathbb{P} * \dot{\mathbb{Q}}$-generic over $V$ if and only if $G_1:=\{p\mid \exists \dot{q}\in\dot{\mathbb{Q}} \ (p,\dot{q})\in G\}$ is $\mathbb{P}$-generic over $V$; and $G_2:=\{\dot{q} \mid \exists p\in G_1 \ (p,\dot{q})\in G\}$, as interpreted in $V[G_1]$, is $\mathbb{Q}$-generic over $V[G_1]$.
  If so, we write $G=G_1 * G_2$.
\end{proposition}

Iteration is better behaved than product:
\begin{proposition}
  If $\mathbb{P}$ is $\tau$-cc and $1_{\mathbb{P}}\forces_{\mathbb{P}} \dot{\mathbb{Q}}\text{ is }\tau\text{-cc}$ then $\mathbb{P} * \dot{\mathbb{Q}}$ is $\tau$-cc; the converse is also true.
  
  The same is also true for $\tau$-closed.
\end{proposition}

Presaturation can be pushed downwards through a two-step iteration:

\begin{lemma}[Lemma 2.12 of \cite{cox_eskew_kill_sat_save_presat}]\label{pushpresatdown}
If $\mathbb{P} * \dot{\mathbb{Q}}$ is $\kappa$-presaturated then $\mathbb{P}$ is $\kappa$-presaturated and $1_\mathbb{P}\forces \dot{\mathbb{Q}}$ is $\kappa$-presaturated.
\end{lemma}

Whether the converse holds is, to the author's knowledge, an open problem; this appears as Question 8.6 of \cite{cox_eskew_kill_sat_save_presat}. 

Longer iterations are possible:
\begin{definition}
  Let $\tau$ be some ordinal. Then a $\tau$-length iteration, written $\seq{\mathbb{P}_\alpha, \dot{\mathbb{Q}}_\alpha\mid \alpha<\tau}$, is a notion of forcing for which:
  \begin{enumerate}
    \item $\mathbb{P}_0\in V$
    \item For all $\alpha>0$, $\dot{\mathbb{Q}}_\alpha$ is a forcing in $V^{\mathbb{P}_\alpha}$ with $P_{\alpha+1}=\mathbb{P}_\alpha * \dot{\mathbb{Q}}_\alpha$
    \item If $\alpha$ is a limit ordinal, then $\mathbb{P}_\alpha$ is some $\alpha$-sequence of $\mathbb{P}_\beta$-terms for $\beta<\alpha$
  \end{enumerate}
\end{definition}

The behavior of a forcing iteration entirely depends on $\mathbb{P}_0$, the $\mathbb{Q}$'s, and what exactly happens at limit stages:
\begin{definition}
  \begin{enumerate}
    \item A forcing iteration is said to be of finite (countable) support if for each $\alpha$ limit, only finitely (countably) many forcing terms drawn from the $\mathbb{P}_\beta$'s, $\beta<\alpha$, are allowed to be nontrivial.
    \item A forcing notion is of direct support if for each $\alpha$ limit, only boundedly many $\mathbb{P}_\beta$-terms below are nontrivial.
    \item A forcing iteration is of inverse support if for each $\alpha$ limit, every term drawn from the $\mathbb{P}_\beta$'s below may be nontrivial.
    \item And a forcing iteration is of Easton support if the iteration is of direct support at regular cardinals, and of inverse support at singular limit ordinals.
  \end{enumerate}
\end{definition}

We will care primarily about Easton support products, and have the following theorem about their preservation of closure and chain condition:

\begin{proposition}
  Any Easton support iteration of $\tau$-closed forcings is $\tau$-closed.
\end{proposition}

\begin{proposition}
  If $\lambda$ is a Mahlo cardinal (i.e. has stationarily many regular, equivalently inaccessible cardinals below), then any $\lambda$-length Easton support iteration of $\lambda$-cc forcings is $\lambda$-cc.
\end{proposition}
