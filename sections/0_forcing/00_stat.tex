\section{Stationary Sets}
\label{forcing:stat}

Solovay's Splitting Theorem for stationary sets accounts for the first major anti-saturation result in set theory.
We will elaborate on the anti-saturation in Section \ref{forcing:ideals_measures} and present the core concepts and behaviors surrounding closed unbounded sets and stationary sets here;
further elaboration can be found in standard references such as \cite{jech_set_theory}, Chapter 8.

\begin{definition}
  Let $\kappa$ be a regular uncountable cardinal.
  Then a $C\subseteq \kappa$ is
  \begin{itemize}
    \item \emph{closed} if whenever $\rho$ is a limit ordinal below $\kappa$ and $C\cap \rho$ is unbounded in $\rho$, then $\rho\in C$
    \item \emph{unbounded} if whenever $\rho<\kappa$ there is some $\rho'\in C\setminus (\rho+1)$ (i.e. some $\rho'>\rho$ such that $\rho'\in C$)
    \item \emph{club} if both closed and unbounded
  \end{itemize}
\end{definition}

The intersection of any two clubs is club, and furthermore:
\begin{proposition}\thlabel{clubs_are_complete}
  Let $\tau<\kappa$ and let $\seq{C_\eta \mid \eta<\tau}$ be a family of club subsets of $\kappa$.
  Then 
  \[
    \bigcap_{\eta<\tau} C_\eta
  \]
  is also club.
\end{proposition}

While the intersection of $\kappa$-many clubs is not necessarily club, there is an approximate version:
\begin{definition}[Diagonal intersection]
  Let $\seq{X_\eta \mid \eta<\lambda}$ be a family of sets of ordinals, with $\lambda$ some ordinal.
  Then their \emph{diagonal intersection} is
  \[
    \diagcap_{\eta<\lambda} X_\eta := \left\{\beta\in Ord \middle| \forall \alpha<\beta \ \beta\in X_\alpha \right\}
  \]
\end{definition}

\begin{proposition}\thlabel{clubs_are_normal}
  Let $\seq{C_\eta \mid \eta<\kappa}$ be a family of club subsets of $\kappa$.
  Then
  \[
    \diagcap_{\eta<\kappa} C_\eta
  \]
  is also club.
\end{proposition}

One may think of the club subsets of $\kappa$ as the ``large" subsets, in a way that will be made precise in Section \ref{forcing:ideals_measures}.
The ``non-small" sets are then the stationary sets, and the ``small" sets nonstationary:
\begin{definition}
  Let $S\subseteq \kappa$. We say $S$ is \emph{stationary} if for every $C\subseteq \kappa$ club, $S\cap C\neq\emptyset$;
  and we say $S$ is \emph{nonstationary} if $S$ is not stationary.
\end{definition}

Since the intersection of two clubs is club, $S\cap C$ is also stationary for any stationary $S$ and club $C$.

Stationary sets admit the following homogeneity result, which, as we will elaborate on in Section \ref{forcing:ideals_measures}, is in a sense equivalent to \thref{clubs_are_normal}:

\begin{theorem}[Fodor's Theorem]
  Let $S$ be stationary and let $f:S\to\kappa$ be regressive (i.e. for every $\alpha\in S$, $f(\alpha)<\alpha$).
  Then there is some $T\subseteq S$ stationary such that $f\upharpoonright T$ is constant.
\end{theorem}

We think of stationary sets as ``non-small" instead of ``large" because any stationary set splits into two disjoint stationary sets.
In fact, more is possible, as shown by Solovay in \cite{solovay_split}:
\begin{theorem}[Solovay's Splitting Theorem]\thlabel{solovay_splitting_theorem}
  For any $S$ a stationary subset of $\kappa$, there exists a family $\seq{T_\alpha \mid \alpha<\kappa}$ of disjoint stationary subsets of $S$ such that
  \[
    S=\bigsqcup_{\alpha<\kappa} T_\alpha
  \]
\end{theorem}

There are analogous notions of club and stationary on more general set algebras:

\begin{definition}
  Let $\kappa$ be a regular cardinal and let $\lambda$ be any cardinal with $\lambda\geq \kappa$.
  We write $\mathcal{P}_\kappa(\lambda)$ to mean the collection of subsets of $\lambda$ of cardinality less than $\kappa$.
  Then:
  \begin{itemize}
    \item A collection $C\subseteq \mathcal{P}_\kappa(\lambda)$ is \emph{closed} if for any $\tau<\kappa$ and any $\subseteq$-increasing chain $\seq{x_\alpha \mid \alpha<\tau}$ of elements of $C$,
    \[
      \bigcup_{\alpha<\tau} x_\alpha
    \]
    is also in $C$
    \item A collection $C\subseteq \mathcal{P}_\kappa(\lambda)$ is \emph{unbounded} if for any $x\in \mathcal{P}_\kappa(\lambda)$ there is some $y\in \mathcal{P}_\kappa(\lambda)$ with $y\supseteq x$ and $y\in C$
    \item If $C$ is both closed and unbounded then $C$ is \emph{club}
    \item And $S\subseteq \mathcal{P}_\kappa(\lambda)$ is \emph{stationary} if $S\cap C\neq\emptyset$ for every club $C$.
  \end{itemize}
\end{definition}

Similar intersection results hold:
\begin{proposition}
  If $\tau<\kappa$ and $\seq{C_\alpha \mid \alpha<\tau}$ is a family of clubs then
  \[
    \bigcap_{\alpha<\tau} C_\alpha
  \]
  is also club.
\end{proposition}

Fodor's Theorem, with a modified notion of $\diagcap$, also holds:
\begin{definition}
  Let $\seq{X_\alpha \mid \alpha<\kappa}$ be a family of subsets of $\mathcal{P}_\kappa(\lambda)$.
  Then
  \[
    \diagcap_{\alpha<\kappa} X_\alpha := \left\{x\in \mathcal{P}_\kappa(\lambda) \middle| x\in \bigcap_{a\in x} X_a\right\}
  \]
\end{definition}

\begin{proposition}
  If $\seq{C_\alpha \mid \alpha<\kappa}$ is a family of club sets on $\mathcal{P}_\kappa(\lambda)$ then
  \[
    \diagcap_{\alpha<\kappa} C_\alpha
  \]
  is also club.
\end{proposition}

\begin{theorem}[Fodor's Theorem]
  If $S$ is a stationary subset of $\mathcal{P}_\kappa(\lambda)$, $f:S\to \mathcal{P}_\kappa(\lambda)$, and $f$ is regressive (i.e. $f(x)\in x$ for every nonempty $x\in S$) then there is a stationary $T\subseteq S$ such that $f\upharpoonright T$ is constant.
\end{theorem}
