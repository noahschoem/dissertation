\section{Ideals, Measures, and Ultrapowers}
\label{forcing:ideals_measures}

We have already seen clubs and nonstationary sets in Section \ref{forcing:stat}.
Their properties generalize to the notions of filter and ideal.
See \cite{jech_set_theory}, Chapter 7, for further elaboration on filters and ideals, and \cite{jech_set_theory}, Chapters 10, 12, and 17 for measurable cardinals and ultrapowers.
We summarize the fundamentals here.

\begin{definition}
  Let $\mathcal{F}$ be a collection of subsets of $\kappa$ (or $\mathcal{P}_\kappa(\lambda)$). Then $\mathcal{F}$ is a \emph{filter} if
  \begin{itemize}
    \item $\kappa$ (or $\mathcal{P}_\kappa(\lambda)$) is in $\mathcal{F}$, and $\emptyset$ is not in $\mathcal{F}$
    \item $\mathcal{F}$ is upwards closed under $\subseteq$
    \item whenever $A$, $B$ are in $\mathcal{F}$ then $A\cap B$ is also in $\mathcal{F}$
  \end{itemize}
  We say $\mathcal{F}$ is \emph{ultra} (equivalently, maximal) if for every $A\subseteq \kappa$ (or $\mathcal{P}_\kappa(\lambda)$), either $A$ is in $\mathcal{F}$, or $A^c$, its complement, is in $\mathcal{F}$.
  
  And we say $\mathcal{F}$ is \emph{nonprincipal} if there is no $X\subseteq \kappa$ (or $\mathcal{P}_\kappa(\lambda)$) such that for all $Y\in\mathcal{F}$, $X\subseteq Y$.
\end{definition}

By the Axiom of Choice, every cardinal and every powerset algebra admits a nonprincipal ultrafilter.

The dual notion of a filter is an \emph{ideal}:

\begin{definition}
  Let $\mathcal{I}$ be a collection of subsets of $\kappa$ (or $\mathcal{P}_\kappa(\lambda)$). Then $\mathcal{I}$ is an \emph{ideal} if
  \begin{itemize}
    \item $\emptyset$ is in $\mathcal{I}$, and $\kappa$ (or $\mathcal{P}_\kappa(\lambda)$) is not in $\mathcal{I}$
    \item $\mathcal{F}$ is downwards closed under $\subseteq$
    \item whenever $A$, $B$ are in $\mathcal{I}$ then $A\cup B$ is also in $\mathcal{I}$
  \end{itemize}
  We say $\mathcal{I}$ is \emph{prime} (equivalently, maximal) if for every $A\subseteq \kappa$ (or $\mathcal{P}_\kappa(\lambda)$), either $A$ is in $\mathcal{I}$, or $A^\complement$, its complement, is in $\mathcal{I}$.
  
  And we say $\mathcal{I}$ is \emph{nonprincipal} if there is no $X\subseteq \kappa$ (or $\mathcal{P}_\kappa(\lambda)$) such that for all $Y\in\mathcal{I}$, $Y\subseteq X$.
\end{definition}

\begin{remark}
  Whenever $\mathcal{F}$ is a filter, its dual ideal is given by 
  \[
    \overset{\smile}{\mathcal{F}}:=\left\{X^\complement \mid X\in \mathcal{F}\right\}
  \]
  Likewise, if $\mathcal{I}$ is an ideal, its dual filter is given by
  \[
    \overset{\smile}{\mathcal{I}}:=\left\{X^\complement \mid X\in \mathcal{I}\right\}
  \]
\end{remark}
For each $\kappa$ (or $\mathcal{P}_\kappa(\lambda)$), the club filter $\mathcal{C}_\kappa$ ($\mathcal{C}_{\mathcal{P}_\kappa(\lambda)}$), consisting of all subsets on $\kappa$ (or $\mathcal{P}_\kappa(\lambda)$) including a club, is a nonprincipal filter.

Likewise, the nonstationary ideal $NS_\kappa$ ($NS_{\mathcal{P}_\kappa(\lambda))}$) consisting of all nonstationary subsets of $\kappa$ (or $\mathcal{P}_\kappa(\lambda)$) is a nonprincipal ideal,
and furthermore, $\overset{\smile}{NS}_\kappa=\mathcal{C}_\kappa$ (and likewise for ${\mathcal{P}_\kappa(\lambda))}$).

\begin{remark}
  The stationary sets we may write as $(NS_\kappa)^+$; in general, whenever $\mathcal{I}$ is an ideal, we write $\mathcal{I}^+$ to denote the collection of all $S$ such that for every $C\in\overset{\smile}{\mathcal{I}}$, $S\cap C\neq\emptyset$.
\end{remark}

Both the nonstationary ideal and the club filter have much stronger closure properties:

\begin{definition}[closure]
  Let $\mathcal{F}$ be a filter and let $\tau$ be a cardinal.
  Then we say that $\mathcal{F}$ is $\tau$-closed if for every $\rho<\tau$ and every family $\seq{A_\alpha \mid \alpha<\rho}$ of sets in $\mathcal{F}$,
  \[
    \bigcap_{\alpha<\rho} A_\alpha
  \]
  is also in $\mathcal{F}$.
  
  Likewise, if $\mathcal{I}$ is an ideal, then $\mathcal{I}$ is $\tau$-closed if for every $\rho<\tau$ and every family $\seq{J_\alpha \mid \alpha<\rho}$ of sets in $\mathcal{I}$,
  \[
    \bigcap_{\alpha<\rho} J_\alpha
  \]
  is also in $\mathcal{I}$.
\end{definition}

Both $\mathcal{C}_\kappa$ ($\mathcal{C}_{\mathcal{P}_\kappa(\lambda)}$) and $NS_\kappa$ ($NS_{\mathcal{P}_\kappa(\lambda))}$) are $\kappa$-closed by \thref{clubs_are_complete}, and by \thref{clubs_are_normal}, are additionally \emph{normal}:

\begin{definition}
  If $\mathcal{F}$ is a filter on $\kappa$ (or $\mathcal{P}_\kappa(\lambda)$, then we say that $\mathcal{F}$ is \emph{normal} if for any $\seq{A_\alpha \mid \alpha<\kappa}$ of sets in $\mathcal{F}$,
  \[
    \diagcap_{\alpha<\kappa} A_\alpha
  \]
  is also in $\mathcal{F}$.
  
  We say that $\mathcal{I}$ is \emph{normal} if its dual filter is normal.
\end{definition}

And Fodor's Theorem generalizes to arbitrary filters and ideals:
\begin{theorem}[Fodor's Theorem]
  Let $\mathcal{I}$ be a $\kappa$-complete ideal.
  Then $\mathcal{I}$ is normal if and only if for every $f:S\to \kappa$ ($\mathcal{P}_\kappa(\lambda)$) regressive with $S\in \mathcal{I}^+$, there is some $T\subseteq S$ with $T\in \mathcal{I}^+$ such that $f\upharpoonright T$ is constant.
\end{theorem}

As we have seen, the existence of nonprincipal ultrafilters (equivalently prime ideals) and $\kappa$-complete normal filters (ideals) are both theorems of $\mathrm{ZFC}$.
Combining the two, however, is a large cardinal hypothesis:

\begin{definition}
  Let $\kappa$ be a cardinal. Then $\kappa$ is \emph{measurable} if there is a normal measure, i.e. $\kappa$-complete normal nonprincipal ultrafilter, $U$ on $\kappa$.
\end{definition}

From a measurable cardinal, we may build an ultrapower of $V$ as follows:

\begin{definition}
  Let $U$ be a normal measure on $\kappa$.
  Then $Ult(V,U)$ is the proper class model of set theory for which:
  \begin{itemize}
    \item the elements are of the form $[f]_U$, the equivalence class of all $f:\kappa \to V$ where $[f]_U=[g]_U$ if for $U$-many $\alpha$, $f(\alpha)=g(\alpha)$, i.e. $\{\alpha<\kappa \mid f(\alpha)=g(\alpha)\}\in U$
    \item and $\in^*$, the binary membership relation, is given by $[f]_U \in^* [g]_U$ if for $U$-many $\alpha$, $f(\alpha)\in g(\alpha)$
  \end{itemize}
\end{definition}

\begin{theorem}[Łoś' Theorem]
  For any formula with parameters $\varphi(\overrightarrow{x})$, for any $\overrightarrow{[f]}$, $Ult(V,U)\models \varphi(\overrightarrow{[f]})$ if and only if for $U$-many $\alpha$, $V\models \varphi(\overrightarrow{f(\alpha)})$.
\end{theorem}

Since $U$ is countably complete, $\in^*$ in $Ult(V,U)$ is well-founded, and therefore, by the Mostowski Collapsing Theorem, its transitive collapse $M$ is a $\in$-model of set theory. 
By Łoś' Theorem, $M$ is also a model of $\mathrm{ZFC}$.

Further, $M$ is a class model since the class map $\rho\mapsto [\alpha\mapsto \rho]_U$ maps to $M$-ordinals and is injective.
By the $\kappa$-completeness, for all $\rho<\kappa$, $[\alpha\mapsto\rho]_U=\rho$; however $[\alpha\mapsto\kappa]_U>\kappa$.
By normality, $\kappa\in M$ since $[\alpha\mapsto\alpha]_U=\kappa$.

There are some additional behavior properties of $M$ that are standard and easy to prove, which we encapsulate in the following equivalent definition of measurable cardinal:

\begin{proposition}
  A cardinal $\kappa$ is measurable if and only if there is a transitive class model $M\subseteq V$ and an elementary map $j:V\to M$ such that 
  \begin{itemize}
    \item $\kappa$ is the least ordinal for which $j(\kappa)>\kappa$
    \item $M$ is closed uner $\kappa$-sequences from $V$, i.e. $V\models M^\kappa \subseteq M$
  \end{itemize}
\end{proposition}

For the converse, if $j:V\to M$ is such an elementary embedding, then $U=\{A\subseteq \kappa \mid \kappa\in j(A)\}$ is a normal measure on $\kappa$.

Even if a nonprincipal filter (e.g. the club filter) is not maximal, we can use forcing to define a form of ultrapower.
We summarize the core ideas and results we need of these generic ultrapowers; see \cite{foreman_ideals_gen_ees} for a more comprehensive overview.

\begin{proposition}
  Let $\mathcal{I}$ be a nonprincipal ideal on $\kappa$.
  Then the poset $\mathcal{B}_{\mathcal{I}}$, defined by
  \[
    \mathcal{B}_{\mathcal{I}}=\left\{[A]_{\overset{\smile}{\mathcal{I}}} \mid A\in \mathcal{I}^+\right\}
  \]
  with partial order $[A]_{\overset{\smile}{\mathcal{I}}} \leq [B]_{\overset{\smile}{\mathcal{I}}}$ if $B\setminus A \in \mathcal{I}$, is a separative notion of forcing.
  Let $G$ be $\mathcal{B}_{\mathcal{I}}$-generic over $V$.
  Then for every $X\in V$, $X\subseteq \kappa$, either $[X]\in G$ or $[X^\complement]\in G$, so $G$ is a $V$-ultrafilter.
  Thus in $V[G]$ we may define $Ult(V,G)$ and obtain an elementary embedding $j:V\to Ult(V,G)$.
  
  Further, if $\mathcal{I}$ is $\kappa$-complete then $G$ is $V$-$\kappa$-complete, and if $\mathcal{I}$ is normal then $G$ is $V$-normal, and $G=\{X\subseteq \kappa \mid \kappa\in j(X)\}$.
\end{proposition}

We will variously refer to the generic ultrapower as $Ult(V,G)$ or $Ult(V,\mathcal{I})$, depending on context,
and will also write $j_I$ to denote $j_G:V\to Ult(V,G)$ the generic ultrapower elementary embedding.

The generic ultrapower is not always well-founded; this is a special property that gets its own name:

\begin{definition}[Definition 2.4 of \cite{foreman_ideals_gen_ees}]
  An ideal $\mathcal{I}$ is said to be \emph{precipitous} if whenever $G$ is a $\mathcal{B}_{\mathcal{I}}$-generic object over $V$,
  $Ult(V,G)$ is well-founded.
\end{definition}

A cardinal $\kappa$ admits a precipitous ideal if and only if $\kappa$ is measurable in some inner model;
however, precipitous ideals need not be prime, and with large cardinals it is possible to force $NS_\kappa$ to be precipitous.

We now turn to saturation, which in certain cases acts as a strong form of precipitousness:

\begin{definition}
  Let $\mu$ be a cardinal and let $\mathcal{I}$ be an ideal. 
  Then we say that $\mathcal{I}$ is \emph{$\mu$-saturated} if $\mathcal{B}_{\mathcal{I}}$ has the $\mu$-chain condition; that is, if for any family $\seq{J_\alpha \mid \alpha<\mu}$ of $\mathcal{I}^+$-sets, there are $\alpha\neq\beta$ such that $J_\alpha \cap J_\beta \in \mathcal{I}^+$.
  
  We say that the \emph{saturation} of $\mathcal{I}$, written $sat(\mathcal{I})$, is the least $\mu$ such that $\mathcal{I}$ is $\mu$-saturated.
\end{definition}

Clearly any ideal on $\kappa$ is $(2^\kappa)^+$-saturated, and a prime ideal is $2$-saturated.

As for why saturation acts as a strong form of precipitousness:
\begin{theorem}
  Let $\mathcal{I}$ be a $\kappa$-complete (nonprincipal) ideal on $\kappa$.
  Then if $sat(\mathcal{I})\leq\kappa^+$, then $\mathcal{I}$ is precipitous.
\end{theorem}

Saturated ideals also lead to a measurable-like closure of the generic ultrapower:

\begin{fact}\label{foreman_ideals_gen_ees:ultrapower-closure}
  If $\mathcal{I}$ is a $\kappa$-complete, $\kappa^+$-saturated ideal in $V$,
  and if $G$ is a $\mathcal{B}_{\mathcal{I}}$-generic filter over $V$,
  then in $V[G]$, $Ult(V,G)^\kappa\subseteq Ult(V,G)$; that is, $Ult(V,G)$ is closed under $\kappa$-sequences from $V[G]$.
\end{fact}

This follows from Propositions 2.9 and 2.14 of \cite{foreman_ideals_gen_ees}.

By the relevant chain condition, we see that if $\mathcal{I}$ is precipitous, then $\mathcal{I}$ also preserves $sat(\mathcal{I})$; this motivates a principle intermediate between saturation and precipitousness:

\begin{definition}
  Let $\mathcal{I}$ be a $\kappa$-complete (nonprincipal) ideal on $\kappa$, and let $\mu$ be a regular cardinal.
  Then $\mathcal{I}$ is $\mu$-presaturated if $\mathcal{I}$ is precipitous and $\mathcal{B}_{\mathcal{I}}$ preserves $\mu$.
\end{definition}

We make the following caveat about presaturation for ideals versus presaturation for posets:

\begin{caveat}
  In contrast to the definition for a general poset in Section \ref{forcing:forcing},
  only for precipitous ideals $I$ is it the case that $I$ is $\mu$-presaturated if and only if $\mathcal{B}_I$ is $\mu$-presaturated as a forcing poset.
  For precipitous ideals, this result appears as Theorem 4.2 of \cite{baumgartner_taylor_sat_gen_ees_two}.
\end{caveat}

Solovay's Splitting theorem can now be rephrased as an anti-saturation result:

\begin{fact}
  For every stationary $S$, $\mathbb{B}_{NS_\kappa}$ is not $\kappa$-saturated below $S$.
\end{fact}

This result can be argued purely using generic ultrapowers:
\begin{proof}[Proof sketch]
  Suppose for sake of contradiction that $NS_\kappa$ is $\kappa$-saturated below $S$.
  Suppose that $S\subseteq Reg\cap \kappa$; then
  \[
    T:=\left\{\alpha \in S \middle| S\cap \alpha \in NS_{\alpha}\right\}
  \]
  is stationary in $\kappa$.
  
  Then a $V$-generic filter $G$ for $\mathbb{B}_{NS_\kappa \upharpoonright T}$ is a $V$-$\kappa$-complete $V$-normal $V$-ultrafilter with well-founded ultrapower $j:V\to Ult(V,G)$ that is closed under $\kappa$-sequences from $V[G]$. 
  Since we forced with $\mathbb{B}_{NS_\kappa \upharpoonright T}$, $T\in G$, hence $\kappa\in j(T)$,
  and so $j(S) \cap \kappa=S$ is no longer stationary in $Ult(V,G)$, hence is nonstationary in $V[G]$. 
  But since $NS_\kappa$ was assumed to be $\kappa$-saturated, our forcing has the $\kappa$-chain condition and hence $S$ must be stationary in $V[G]$; this is a contradiction.
  
  A similar argument holds if $S \cap \cof(\lambda)$ is stationary for some $\lambda<\kappa$; for if we force in $\mathbb{B}_{NS_\kappa}$ below $S \cap \cof(\lambda)$, we get that $\kappa\in j(S\cap \cof(\lambda))$ and thus $\cf^{Ult(V,G)}(\kappa)=\lambda$.
  But $\mathbb{B}_{NS_\kappa}$ is $\kappa$-cc, so preserves the regularity of $\kappa$; this is a contradiction.
\end{proof}

However, there are still useful arguments that can be written \emph{just} from having that $\mathbb{B}_{NS_\kappa}$ is precipitous.
For example, this simplifies Silver's original argument in \cite{silver_sch} that if $SCH$ fails at a singular cardinal, then the first singular cardinal at which $SCH$ fails must have countable cofinality.
Precipitousness results concerning nonstationary ideals have a long history, and are well collated in \cite{cummings_itd_forcing_ees}) and  \cite{foreman_ideals_gen_ees}.

To compute saturation (and other) properties on ideals in generic extensions, we may use Foreman's Duality Theorems.
Let $\mathbb{P}$ be a notion of forcing and let $\mathcal{I}$ be a precipitous ideal in $V$;
we write $\overline{\mathcal{I}}$ to denote $\{A\in V^{\mathbb{P}} \mid \exists X\in \mathcal{I} \ A\subseteq X\}$.
Then Foreman's Duality Theorems (the many forms of which can be found in \cite{foreman_ideals_gen_ees}, Chapter 7.4) allow for the computation of various (including saturation) properties of $\overline{\mathcal{I}}$, roughly as follows:

\begin{theorem}
  Under certain assumptions, there exist conditions $R\in \mathbb{P} * \dot{\overline{\mathcal{I}}}$ and $S\in \mathcal{I} * \dot{j_{\mathcal{I}}(\mathbb{P})}$ such that
  \[
    \left(\mathbb{P} * \dot{\overline{\mathcal{I}}}\right)/R \cong
    \left(\mathcal{I} * \dot{j_{\mathcal{I}}(\mathbb{P})}\right)/S
  \]
\end{theorem}
