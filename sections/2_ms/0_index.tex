\chapter{Mutual Stationarity}
\label{ms}

Foreman and Magidor first introduced mutual stationarity in \cite{foreman_magidor_ns_ns} to argue for the non-saturation of certain nonstationary ideals.
Mutual stationarity generalizes stationarity to singular cardinals, and leads to interesting avenues of study in its own right.
We will first define mutual stationarity and summarize prior results, with additional commentary and detail.
This section culminates in the following mutual stationarity property for Magidor generic sequences: that if $\seq{\kappa_\alpha \mid \alpha<\lambda}$ is a Magidor generic sequence for some $\lambda<\kappa$ over some $V$, then there is a cofinite subset of the Magidor generic sequence $K$ such that any family $S_\alpha\subseteq\kappa_\alpha\in K$ of statioary sets is mutually stationary.
We spell this out in Section \ref{ms:magidor_gens}, with the core results being \thref{magidor_generics_ms_cobounded} and \thref{magidor_generics_ms_cofinite}.

\section{Background}
\label{ms:ms}

To see that mutual stationarity generalizes stationarity, we recall that the following property precisely characterizes stationarity:

\begin{proposition}\thlabel{stationarity_as_elementarity}
  Let $\kappa$ be a regular cardinal, and let $S\subseteq \kappa$.
  Then $S$ is stationary if and only if whenever $\mathcal{U}$ is an algebra on $\kappa$ (equivalently on some $\lambda\geq \kappa$), there is a $\mathcal{B}\prec \mathcal{U}$ such that $\sup(\mathcal{B}\cap \kappa)\in S$.
\end{proposition}

With this in mind, we may generalize this to sets on larger sequences of cardinals:

\begin{definition}\thlabel{definition_mutual_stationarity}
  Suppose $K:=\seq{\kappa_\alpha \mid \alpha<\mu}$ is an increasing sequence of regular cardinals with supremum $\lambda$ of cofinality $\mu$.
  
  We say that a family $\seq{S_\alpha\subseteq\kappa_\alpha \mid \alpha<\mu}$ is \emph{mutually stationary} if whenever $\mathcal{U}$ is an algebra on $\lambda$, there is a $\mathcal{B}\prec \mathcal{U}$ such that for all $\alpha\in\mathcal{B}\cap K$, $\sup(\mathcal{B}\cap \kappa_\alpha)\in S_\alpha$. (Note that any mutually stationary sequence of sets is necessarily composed of stationary sets.)
  
  We say $MS(\kappa_\alpha \mid \alpha<\mu)$ holds if every family $\seq{S_\alpha\subseteq\kappa_\alpha \mid \alpha<\mu}$ of stationary sets is mutually stationary.
  
  If we restrict our attention to stationary sets of cofinality $\delta$ above, we analogously write $MS(\seq{\kappa_\alpha \mid \alpha<\mu},cof(\delta))$.
\end{definition}

By \thref{stationarity_as_elementarity}, any mutually stationary family consists of stationary sets; and the questions of whether there \emph{is} a mutually stationary family satisfying certain conditions, and whether any amount of $MS(\circ)$, holds, are both independently interesting questions.
We mostly focus on the consistency of $MS(\circ)$-principles in this chapter.

Many mutual stationarity proofs are analogous to, or generalizations of, the following:

\begin{theorem}\thlabel{ms_measurables}
  Let $\seq{\kappa_\alpha \mid \alpha<\lambda, \alpha\text{ successor}}$ be an increasing sequence of measurable cardinals with limit $\kappa$, and with $\lambda<\kappa$.
  
  Then $MS(\seq{\kappa_\alpha \mid \alpha<\lambda, \alpha\text{ successor}})$ holds.
\end{theorem}

Foreman and Magidor state this without proof in \cite{foreman_magidor_ns_ns}, but the proof is fairly straightforward, and proceeds via the use of \emph{indiscernibles}:

\begin{definition}
  Let $\mathcal{M}$ be a structure, $I$ a linear order.
  Then a sequence $\seq{a_i\mid i\in I}\subseteq\mathcal{M}$ is \emph{indiscernible}\footnote{
    Some authors use the term \emph{order-indiscernible} instead for this concept.
  }if for every formula $\phi$ in the signature of $\mathcal{M}$ and every $i_1<\dots<i_n$ and $j_1<\dots<j_n$,
  \[
    \mathcal{M} \models \phi\left(a_{i_1},\dots,a_{i_n}\right)
    \iff
    \mathcal{M}\models \phi\left(a_{j_1},\dots,a_{j_n}\right)
  \]
\end{definition}

\begin{proof}[Proof of \thref{ms_measurables}]
  Let $\mathcal{A}$ be an algebra on $\kappa$; augment $\mathcal{A}$ to $\mathcal{A}'$ with terms for each $S_\alpha$.
  For each $\alpha$, let $H_\alpha$ be a measure one (with respect to a normal measure on $\kappa_\alpha$) family of $\mathcal{A}$-order-indiscernibles for $\mathcal{A}\cap \kappa_\alpha$.
  Since $Lim(H_\alpha)$, the collection of limit points of $H_\alpha$, is then a club, let $\gamma_\alpha\in S_\alpha\cap Lim(H_\alpha)$, let $H'_\alpha=H_\alpha \cap \gamma_\alpha$, 
  and let $\mathcal{M}=Hull_{\mathcal{A}'}\left(\bigcup_{\alpha<\lambda}H'_\alpha \mid \alpha<\lambda\right)$.
  
  Clearly for each $\alpha$, $\mathcal{M}\cap \kappa_\alpha \geq \gamma_\alpha$.
  To see that this is an equality, suppose that for some Skolem term $F$ and some $x\in \left[\bigcup_{\alpha<\lambda}H'_\alpha \mid \alpha<\lambda\right]^{<\omega}$, $F(x)=\beta$; let $\alpha$ be a successor ordinal such that $\beta\in [\kappa_{\alpha-1}, \kappa_{\alpha})$.
  
  Since $H_{\alpha}$ is unbounded in $\kappa_\alpha$, let $\delta \in H_\alpha \cap (\max\{\gamma_\alpha,\beta\},\kappa_{\alpha})$.
  Since $x$ is finite and $H_\alpha$ is unbounded in $\gamma_\alpha$, let $\zeta\in H_\alpha \cap (\max(x\cap \gamma_\alpha), \gamma_\alpha)$.
  
  By choice of $\delta$, we have that $\mathcal{A'}\models F(x)<\delta$; but by how we chose $\delta$ and $\zeta$, and since $x\cap [\zeta,\kappa_\alpha)=\emptyset$, the elements of $x$ bear the exact same order relation to $\zeta$ as they do to $\delta$.
  \footnote{
    In time, we will give this a precise name; see \thref{koepke_types} in \ref{ms:koepke}.
  }
  Therefore, since $\mathcal{M}\models F(x)<\delta$ and $x$, $\delta$, $\zeta$ are order indiscernibles for $\mathcal{M}$, we have that $\mathcal{M}\models F(x)<\zeta$ and $\zeta<\gamma_\alpha$.
\end{proof}

A largely similar argument works for the points of a Prikry generic sequence:

\begin{theorem}[Cummings, Foreman, and Magidor, Theorem 5.4 of \cite{cummings_foreman_magidor_canonical_structure}]\thlabel{ms_prikry}
  Let $\kappa$ be measurable with normal measure $U$, let $\mathbb{P}(U)$ be Prikry forcing at $\kappa$, and let $\seq{\kappa_n \mid n<\omega}$ be Prikry generic.
  Then in the generic extension, there is some $m<\omega$ such that 
  \[
    MS(\seq{\kappa_n \mid m\leq n<\omega})
  \]
  holds.
\end{theorem}

This theorem is due to Cummings, Foreman, and Magidor, and its  proof originally appeared in \cite{cummings_foreman_magidor_canonical_structure}.
But, in ways we will see in \ref{ms:koepke} was anticipated much earlier by results of Koepke in \cite{koepke_free_subset}.

\begin{proof}[Proof sketch]
  With the use of the Prikry property, the argument of \thref{ms_measurables} readily adapts to a $U$-measure one 
  set $B$ of order indiscernibles below $\kappa$;
  then $(\emptyset,B\cap Lim(B))$ forces the desired mutual stationarity property.
  We will further refine and elaborate on this argument in \ref{ms:koepke}.
\end{proof}

The following lemmas make working with mutually stationary sequences and mutual stationarity principles easier.

\begin{lemma}[Folklore]
  \begin{enumerate}
    \item Any subsequence of a mutually stationary family of sets is also mutually stationary.
    \item If $\seq{S_\alpha\mid \alpha<\mu}$ is mutually stationary with each $S_\alpha\subseteq \kappa_\alpha\cap cof(\leq\nu)$ and $\nu<\kappa_0$ then the mutual stationarity can be witnessed by an elementary substructure of size $\nu$.
  \end{enumerate}
\end{lemma}
Foreman and Magidor in \cite{foreman_magidor_ns_ns} cite this as implicit in Baumgartner's paper \cite{baumgartner_size_clubs}.

As a corollary, mutual stationarity principles of restricted cofinality (under mild assumptions) are preserved under finite changes:

\begin{lemma}[Foreman and Magidor, Lemma 23 of \cite{foreman_magidor_ns_ns}]\thlabel{mutual_stationarity_finite_change}
  Let $\seq{S_\alpha \mid \alpha<\mu}$ be mutually stationary with each $S_\alpha \subseteq \kappa_\alpha \cap cof(\leq\nu)$, with $\nu<\kappa_0$.
  Suppose $\lambda_0,\dots,\lambda_n$ are all greater than $\nu$, not equal to any $\kappa_\alpha$, and $S_{\lambda_i}\subseteq \lambda_i\cap cof(\leq\nu)$.
  
  Then 
  \[
    \seq{S_\alpha \mid \alpha<\mu} \frown \seq{S_{\lambda_i} \mid i<n}
  \]
  is also mutually stationary.
\end{lemma}

As for consistency results:

\begin{theorem}[Foreman and Magidor, Theorem 7 of \cite{foreman_magidor_ns_ns}]
  Let $\seq{\kappa_\alpha \mid \alpha<\mu}$ be any family of regular cardinals.
  Then 
  \[
    MS(\seq{\kappa_\alpha \mid \alpha<\mu},cof(\omega))
  \]
\end{theorem}

\begin{theorem}[Foreman and Magidor, Theorem 24 of \cite{foreman_magidor_ns_ns}]
  In $L$, for all $k\in\omega$, 
  \[
    \lnot MS(\seq{\aleph_n \mid n>k},cof(\omega_k))
  \]
\end{theorem}

Further inner model theoretic results showed that mutual stationarity principles for fixed uncountable cofinality requires large cardinals incompatible with $L$:

\begin{theorem}[Theorem 1.4 and Corollary 1.5 of \cite{koepke_welch_global_square_ms}]
  If $k<\omega$ and $MS(\seq{\aleph_n \mid n>k},cof(\omega_k))$, then there is an inner model in which each $\aleph_n^V$ has stationarily many measurable cardinals of Mitchell order $\omega_{n-2}$.
\end{theorem}

As of writing, the best known argument in the other direction comes courtesy of Ben-Neria in 2019:

\begin{theorem}[Ben-Neria, Theorem 1.3 of \cite{benneria_singular_stat}]
  Assume $GCH$ and let $\seq{\kappa_n\mid n<\omega}$ be a sequence of cardinals with $\kappa_0=\omega$, and with limit $\kappa_\omega$ such that each $\kappa_n$ is $\kappa_\omega^+$-supercompact.
  Then after forcing with a full-support iteration of $Col(\kappa_n,<\kappa_{n+1})$, for each $k<\omega$, 
  \[
    MS(\seq{\aleph_n\mid n>k},cof(\omega_k))
  \]
  holds.
\end{theorem}

Mutual stationarity results at every other $\aleph_n$ require much weaker large cardinal assumptions.
For instance:

\begin{theorem}[Koepke, Theorem 1.6 of \cite{koepke_ms_omega_1}; Koepke and Welch, Theorem 1 of \cite{koepke_welch_strength_ms}]
  The principle 
  \[
    MS(\seq{\aleph_{2n+3}\mid n<\omega},cof(\omega_1))
  \]
  and the existence of a measurable cardinal are equiconsistent.
\end{theorem}

and for larger cofinality:
\begin{theorem}[Ben-Neria, Theorem 1.4 of \cite{benneria_singular_stat}]
  It is consistent, relative to a sequence $\seq{\kappa_n\mid n<\omega}$ with each $\kappa_n$ being $\kappa_n^+$-supercompact with a $\kappa_{n-1}^+$-Mitchell sequence of such measures, that every sequence $S_n\subseteq \aleph_{2n+3}\cap cof(<\aleph_{2n+2})$ of stationary subsets is mutually stationary.
  
  In particular, for all $k<\omega$,
  \[
    MS(\seq{\aleph_{2n+3}\mid k<2n+3<\omega},cof(\omega_k))
  \]
\end{theorem}

This family of consistency results demonstrate that $MS$ principles has large cardinal strength, often follows directly from large cardinals and, failing that, in a forcing extension, and fails in $L$.
However, relationships between square principles and mutual stationarity are not straightforward.
Koepke and Welch in \cite{koepke_welch_global_square_ms} use global $\square$ in the Dodd-Jensen core model $K$ to construct inner models for large cardinals from $MS$ principles, but relationships between $MS$ principles and failures of $\square$ remain uninvestigated.

We conclude our background section by elaborating on how we may view mutual stationarity principles as a strong form of anti-saturation.

Its introduction implicitly takes this view in the form of the following (a weaker version of which appears as Theorem 7 and Corollary 8 in Foreman and Magidor's \cite{foreman_magidor_ns_ns}):

\begin{theorem}\thlabel{ms_antisat}
  Assume $\delta$ is the successor of a regular cardinal and $\lambda$ is a singular cardinal with $cf(\lambda)=\mu>\delta$.
  Let $\seq{\kappa_\alpha\mid \alpha<\mu}$ be a cofinal sequence of cardinals in $\lambda$ and suppose $MS(\seq{\kappa_\alpha\mid \alpha<\mu},cof(<\delta))$.
  Then the nonstationary ideal on $\mathcal{P}_\delta(\lambda)$ is not $\lambda^\mu$-saturated.
\end{theorem}

Under $\lnot SCH_\lambda$ with $\lambda$ of uncountable cofinality, this would give a more impressive anti-saturation result for the nonstationary ideal on $(\mathcal{P}_\delta(\lambda))$ than previously known.
Thus compatibility of mutual stationarity principles with $\lnot SCH$ are of independent interest.

\begin{proof}[Proof of \thref{ms_antisat}]
  Let $\seq{S_\alpha\subseteq\kappa_\alpha \cap cof(<\delta) \mid \alpha<\mu}$ 
  be a sequence of stationary sets.
  By Solovay's Splitting theorem, for each $\alpha$ let $\seq{T^\alpha_\beta\mid \beta<\kappa_\alpha}$ be a pairwise disjoint sequence of stationary subsets of $S_\alpha$.
  
  For each $f\in \prod_{\alpha<\mu} \kappa_\alpha$, let $S_f=\{M\in \mathcal{P}_\delta(\lambda) \mid \alpha\in M\implies \sup(M\cap \kappa_\alpha)\in T^\alpha_{f(\alpha)}\}$.
  
  Since $\seq{T^\alpha_\beta\mid \beta<\kappa_\alpha}$ is mutually stationary, each $S_f$ is stationary in $\mathcal{P}_\delta(\lambda)$.
  Further, if $f\neq g$, then $f(\alpha)\neq g(\alpha)$ and hence $S^\alpha_{f(\alpha)}\cap S^\alpha_{g(\alpha)}=\emptyset$.
  Thus $S_f\cap S_g$ misses the club $\{M\in \mathcal{P}_\delta(\lambda)\mid \alpha\in M\}$.
\end{proof}

However, very little is known about mutual stationarity principles with $\lnot SCH$.

Koepke's argument in \cite{koepke_ms_omega_1} does not depend on cardinal arithmetic, so should readily achieve $MS(\seq{\aleph_{2n+3}\mid n<\omega; cof(\leq\omega_1)}) + \lnot SCH_{\aleph_\omega})$.

Ben-Neria's arguments in \cite{benneria_singular_stat} assume $GCH$, but suggest a framework for achieving mutual stationarity principles with the failure of $GCH$.

% \section{Mutual Stationarity as Antisaturation Result}
\label{ms:antisaturation}

In line with mutual stationarity's original purpose, we may view mutual stationarity principles as a strong form of antisaturation.

Its introduction implicitly takes this view in the form of the following (a weaker version of which appears as Theorem 7 and Corollary 8 in \cite{foreman_magidor_ns_ns}):

\begin{theorem}\thlabel{ms_antisat}
  Assume $\delta$ is the successor of a regular cardinal and $\lambda$ is a singular cardinal with $cf(\lambda)=\mu>\delta$.
  Let $\seq{\kappa_\alpha\mid \alpha<\mu}$ be a cofinal sequence of cardinals in $\lambda$ and suppose $MS(\seq{\kappa_\alpha\mid \alpha<\mu},cof(<\delta))$.
  Then the nonstationary ideal on $\mathcal{P}_\delta(\lambda)$ is not $\lambda^\mu$-saturated.
\end{theorem}

Under $\lnot SCH_\lambda$ with $\lambda$ of uncountable cofinality, this would give a more impressive antisaturation result for the nonstationary ideal on $(\mathcal{P}_\delta(\lambda))$ than previously known.
Thus compatibility of mutual stationarity principles with $\lnot SCH$ are of independent interest.

\begin{proof}[Proof of \thref{ms_antisat}]
  Let $\seq{S_\alpha\subseteq\kappa_\alpha \cap cof(<\delta) \mid \alpha<\mu}$ 
  be a sequence of stationary sets.
  By Solovay's Splitting theorem, for each $\alpha$ let $\seq{T^\alpha_\beta\mid \beta<\kappa_\alpha}$ be a pairwise disjoint sequence of stationary subsets of $S_\alpha$.
  
  For each $f\in \prod_{\alpha<\mu} \kappa_\alpha$, let $S_f=\{M\in \mathcal{P}_\delta(\lambda) \mid \alpha\in M\implies \sup(M\cap \kappa_\alpha)\in T^\alpha_{f(\alpha)}\}$.
  
  Since $\seq{T^\alpha_\beta\mid \beta<\kappa_\alpha}$ is mutually stationary, each $S_f$ is stationary in $\mathcal{P}_\delta(\lambda)$.
  Further, if $f\neq g$, then $f(\alpha)\neq g(\alpha)$ and hence $S^\alpha_{f(\alpha)}\cap S^\alpha_{g(\alpha)}=\emptyset$.
  Thus $S_f\cap S_g$ misses the club $\{M\in \mathcal{P}_\delta(\lambda)\mid \alpha\in M\}$.
\end{proof}

However, very little is known about mutual stationarity principles with $\lnot SCH$.

Koepke's argument in \cite{koepke_ms_omega_1} does not depend on cardinal arithmetic, so should readily achieve $MS(\seq{\aleph_{2n+3}\mid n<\omega; cof(\leq\omega_1)}) + \lnot SCH_{\aleph_\omega})$.

Ben-Neria's arguments in \cite{benneria_singular_stat} assume $GCH$, but suggest a framework for achieving mutual stationarity principles with the failure of $GCH$.

\section{The Koepke Approach}
\label{ms:koepke}

Koepke's argument for the consistency of $MS(\seq{\aleph_{2n+3}\mid n<\omega;\omega_1})$ uses an alternative way to think about the proof of \thref{ms_prikry}.
We will flesh out this approach here, as our arguments for Magidor generics depend on Koepke's work.

Koepke's mutual stationarity result in \cite{koepke_ms_omega_1} implicitly rephrases \thref{ms_prikry} as a Ramsey-theoretic result, in the following sense:

\begin{definition}[Definition 2.2 of \cite{koepke_ms_omega_1}]\thlabel{koepke_types}
   Suppose that $\kappa$ is a limit cardinal with $cf(\kappa)=\lambda$, and let $\seq{\kappa_\alpha \mid \alpha<\lambda}$ be a normal cofinal sequence of cardinals below $\kappa$.
  \begin{enumerate}
     \item Let $x\in \left[\kappa\right]^{<\omega}$. Then $type(x)=\seq{\left|x \cap \kappa_\alpha\right| \mid \alpha\in \lambda}$. 
    
     We'll say that $t\in [\omega]^\lambda$ is a type if for some $x$, $t=type(x)$.
    
     We'll additionally say that the \emph{instances} of $t$ are $inst(t) := \{x\in [\kappa]^{<\omega} \mid type(x) = t\}$
    
    % probably don't need projection
    \begin{comment}
    \item Let $a\subseteq \omega$.
    Then the \emph{projection by $a$}, $\upharpoonright a:[\kappa]^{<\omega} \to [\kappa]^{<\omega}$, is defined by 
    \[
      x\upharpoonright a = \{\xi \in x \mid o.t.(\xi\cap x)\in a\}
    \]
    
    \item Because of how types are defined, for all $a$, if $type(x)=type(y)$ then $type(x\upharpoonright a)=type(y\upharpoonright a)$.
    So if $t$ is a type, let $t\upharpoonright a$ be $type(x\upharpoonright a)$ for any $type(x)=t$.
    \end{comment}
    
    \item We say that a sequence of sets $\seq{I_\alpha \mid \alpha\in\lambda\cap Succ}$ with each $I_\alpha \subseteq \kappa_\alpha$ is \emph{mutually homogeneous for a function $F:[\kappa]^{<\omega}\to \kappa$} if for every $x,y\in\left[\bigcup_{\alpha \in \lambda\cap Succ} I_\alpha\right]^{<\omega}$, we have that $F(x)=F(y)$ whenever $type(x)=type(y)$ and $x\cap (F(x)+1) = y \cap (F(x) + 1)$.
    
    \item We say that $\seq{\kappa_\alpha \mid \alpha<\lambda}$ is \emph{mutually Ramsey} if for all $F:[\kappa]^{<\omega} \to \kappa$, there is a mutually homogeneous for $F$ sequence $\seq{I_\alpha \mid \alpha\in\lambda\cap Succ}$ such that $|I_\alpha|=\kappa_\alpha$.
  \end{enumerate}
\end{definition}

\begin{remark}
  \begin{enumerate}
    \item For $x\in[\kappa]^{<\omega}$, $type(x)$ is the (non-decreasing, eventually stabilizing) sequence that tells you, at point $\alpha$, how many elements of $x$ there are below $\kappa_\alpha$.
    
    % projection temporarily removed
    % \item Projection captures a notion of subsequence; for instance, if $x=\seq{\kappa_0,\kappa_1,\dots,\kappa_{20}}$ and $a$ is the even numbers, then $x\upharpoonright a=\seq{\kappa_0,\kappa_2,\kappa_4,\dots,\kappa_{18},\kappa_{20}}$, and e.g. $\kappa_1\notin x\upharpoonright a$ because $o.t.(x\cap \kappa_1)=o.t.(\{\kappa_0\})=1\notin a$.
    
    % Based on this observation, we may equivalently say that if $x=\seq{x_i \mid 0\leq i<n}$ written in increasing order, then $x \upharpoonright a=\{x_j \mid 0\leq j<n, \ j\in a\}$.
    
    \item In the definition of mutually homogeneous, the restriction $x\cap (F(x)+1) = y \cap (F(x) + 1)$ is equivalent to $F(x)<\min(x \Delta y)$ (the symmetric difference).
    
    The idea here is that a sequence is mutually homogeneous for $F$ if $F$'s values on the (finite subsets picked from the) sequence depend only on type and a generalization of regressiveness. In a sense, this is exactly the behavior one would want out of a Skolem function on a structure with many large order-indiscernibles, which is exactly how \thref{ms_measurables} and \thref{ms_prikry} proceed.
  \end{enumerate}
\end{remark}

% more projection stuff commented out
\begin{comment}
It may be more natural to think about types as follows:

\begin{definition}
  Let $t$ be a type, and write $t=\seq{b_\alpha \mid \alpha<\lambda}$.
  Then the \emph{signature of} $t$ is $s = sig(t) = \seq{b_\alpha - b_{\alpha-1} \mid \alpha\in (\lambda \cap Succ) \cup \{0\}}$ (where $b_{-1}:=0$).
\end{definition}

Type is meant to say ``below $\kappa_\alpha$, we have seen $b_\alpha$-many points" whereas signature says, ``there are $b_\alpha - b_{\alpha-1}$-many points in $[\kappa_{\alpha-1},\kappa_\alpha)$".

Observe that given $s\in \omega^{<\lambda}$, if the support of $s$ is finite then we can precisely identify the type $t$ with signature $s$ by $b_\alpha = \sum_{i=0}^\alpha s(i)$.

Signature allows us further characterizations of projections.
If you want to project from $t'$ to $t$, then for all $\alpha$, $t'$ must have at least as many points in $[\kappa_{\alpha-1},\kappa_\alpha)$ as $t$.
Thus we have the following observation:

\begin{remark}
  There is an $a\subseteq \omega$ such that $t' \upharpoonright a = t$ iff for all $\alpha$, $sig(t')(\alpha) \geq sig(t)(\alpha)$.
\end{remark}

Let's say $t'$ expects $4$ points in $[\kappa_{\alpha-1},\kappa_\alpha)$ and $t$ expects $2$. Then if we can project from $t'$ to $t$, then within $[\kappa_{\alpha-1},\kappa_\alpha)$ we have $6$ different possibilities of which indices within $[\kappa_{\alpha-1},\kappa_\alpha)$ we'll pick when we project, coming from choosing $4$ of the $2$ indices. 
This leads to the following remark:

\begin{remark} \thlabel{finitely_many_projections}
  If we can in fact project from $t'$ to $t$, then the number of distinct maps $(\upharpoonright a) \upharpoonright inst(t') : inst(t') \to inst(t)$ is
  \[
    \prod_{\alpha\in (\lambda \cap Succ)} \binom{sig(t')(\alpha)}{sig(t)(\alpha)}
  \]
  where for all but finitely many $\alpha$, $\binom{sig(t')(\alpha)}{sig(t)(\alpha)} = 1$. In particular, the number of such maps $\upharpoonright a$ is finite.
\end{remark}
\end{comment}

Our use case for mutually homogeneous sequences is to build order indiscernibles for a given model $\mathcal{M}$:

\begin{definition}\thlabel{mutually_homogeneous_for_structure}
  Let $\theta\geq \kappa$, and let $\mathcal{M}$ be a model of domain $V_\theta$ (or $H_\theta$ or $\theta$) with signature of size less than $\kappa_0$.
  Suppose that $\mathcal{M}$ has a term $F$ which is a Skolem function for $\mathcal{M}$ (i.e. $F:[dom(\mathcal{M})]^{<\omega}\to \theta$ and for every $X\subseteq \kappa$, $X\subseteq F"[X]^{<\omega}\prec \mathcal{M}$).
  Then we say that $\mathcal{I}:=\seq{I_\alpha \mid \alpha\in \lambda\cap Succ}$ is \emph{mutually homogeneous for $\mathcal{M}$ and $F$} if
  \begin{itemize}
    \item $\mathcal{I}$ is mutually homogeneous for $F$
    \item for each formula $\phi$ in the signature of $\mathcal{M}$, if $x,y\in [\kappa]^{<\omega}$ are of the appropriate arity for $\phi$ and $type(x)=type(y)$, then $\mathcal{M}\models \phi(x) \iff \mathcal{M}\models \phi(y)$
    \item for each term definable in $\mathcal{M}$ viewed as a map $g:[\kappa]^{<\omega}\to \kappa$, $\mathcal{I}$ is mutually homogeneous for $g$; that is, if $x,y\in \left[\bigcup_{\alpha \in \lambda\cap Succ} I_\alpha\right]^{<\omega}$ with $type(x)=type(y)$ and $x\cap (g(x)+1) = y \cap (g(x)+1)$ then $g(x)=g(y)$
  \end{itemize}
\end{definition}

\begin{remark}\thlabel{rmk_construct_mh_for_structure}
  If $\seq{\kappa_\alpha \mid \alpha\in \lambda \cap Succ}$ is mutually Ramsey such that each $I_\alpha\subseteq \kappa_\alpha$ may be chosen to be club (or measure one for some appropriate notion of ``measure one") in $\kappa_\alpha$,
  then constructing a mutually homogeneous system for $\mathcal{M}$ is straightforward, albeit with the following adjustment needed for the formulas $\phi$:
  
  We may view each formula $\phi$ restricted to parameters chosen from $[\kappa]^{<\omega}$ as a function $\phi:[\kappa]^{<\omega} \to \{0,1,2\}$ defined by: 
  \[
    \phi(x) = \begin{cases}
      0 & |x|\text{ is inappropriate for use in }\phi \\
      1 & \mathcal{M} \models \lnot \phi(x) \\
      2 & \mathcal{M} \models \phi(x) \\
    \end{cases}
  \]
  Then as long as we have that $\seq{I_\alpha \mid \alpha\in \lambda \cap Succ}$ is mutually homogeneous for $\phi$ and that $0,1,2\notin I_0$, we have that $y\cap (\phi(x)+1)=\emptyset$ for every $x,y\in [\kappa]^{<\omega}$; therefore, if $type(x)=type(y)$ then $\phi(x)=\phi(y)$.
  
  So to obtain a mutually homogeneous system for $\mathcal{M}$, we construct mutually homogeneous systems of clubs (or measure one sets) for $F$ and for each $\phi$ and each $g$ as in \thref{mutually_homogeneous_for_structure}, of which there will be less than $\kappa_0$-many, and take their intersection.
\end{remark}

\thref{ms_prikry}, the result on Prikry generics in \cite{cummings_foreman_magidor_canonical_structure} fundamentally relies only on arguments involving such $F:[\kappa]^{<\omega}\to \kappa$. 
Koepke has arguments pertaining to such $F$ and in a much earlier paper \cite{koepke_free_subset} proved the following:

\begin{theorem}[\cite{koepke_free_subset}; also Proposition 2.3 of \cite{koepke_ms_omega_1}]\thlabel{koepke_mr}
  Let $V[G]$ be a Prikry generic model for $\kappa$ over $V$, and let $\seq{\kappa_n \mid n<\omega}$ be a Prikry generic sequence for $\kappa$.
  Then in $V[G]$, for some $m$, $\seq{\kappa_n \mid m\leq n<\omega}$ is mutually Ramsey.
  Furthermore, the sequence of $I_n$'s witnessing this in $V[G]$ may be chosen to be club subsets of each $\kappa_n$, or stationary subsets of each $\kappa_n$ consisting only of inaccessibles.
\end{theorem}

The proof is fundamentally exactly as in \cite{cummings_foreman_magidor_canonical_structure} but Koepke reproves it anyways.
The clubness is left unwritten in both, but is implicit; the fact that each $I_n$ may consist only of inaccessibles comes from observing that measure-one (hence stationarily) many cardinals below a measurable are Mahlo.

\begin{corollary}
  If $\seq{\kappa_n\mid n<\omega}$ is a Prikry generic sequence over $V$ in $V[G]$,  then in $V[G]$, $MS(\seq{\kappa_n\mid n<\omega})$ holds on a tail.
\end{corollary}

\begin{proof}
  Let $\vec{S}=\seq{S_n\mid n<\omega}$ be a family of sationary sets on $\seq{\kappa_n\mid n<\omega}$; without loss of generality, all of $\seq{\kappa_n\mid n<\omega}$ is mutually Ramsey.
  
  Let $\mathcal{M}$ be a structure on $\kappa$ with Skolem function $F$ and, without loss of generality, terms for each $S_n$; by mutual Ramseyness of $\kappa$, let $\seq{I_n\mid n<\omega}$ be a mutually homogeneous for $\mathcal{M}$ and $F$ family of clubs on each $\kappa_n$.
  
  Since each $I_n$ is unbounded in $\kappa_n$, let $\gamma_n\in S_n\cap Lim(I_n)$, and let 
  \[
    \mathcal{U}=F"\left[\bigcup_{n<\omega} I_n \cap \gamma_n\right]^{<\omega}
  \]
  
  Since $F$ is a Skolem function and since $\gamma_n \in Lim(I_n)$, $\mathcal{U}\prec \mathcal{M}$ and $\sup(\mathcal{U}\cap \kappa_n)\geq \gamma_n$. 
  
  To see that $\sup(\mathcal{U}\cap \kappa_n)\leq \gamma_n$, let $x\in[\bigcup_{n<\omega} I_n \cap \gamma_n]^{<\omega}$ be such that $F(x)=\alpha\in [\kappa_{n-1},\kappa_n)$ for some $n$.
  
  Since $I_n$ is unbounded in $\kappa_n$, let $\delta\in I_n \cap (\max\{\gamma_n,\alpha\},\kappa_n)$;
  since $x$ is finite and $I_n$ is unbounded in $\gamma_n$, let $\zeta\in I_n \cap (\max (x\cap \gamma_n),\gamma_n)$.

  We now have that $\mathcal{M}\models F(x)<\delta$ by choice of $\delta$.
  Additionally, by choice of $\delta$ and $\zeta$, and since $x\cap [\zeta,\kappa_n)=\emptyset$, the elements of $x$ bear the same order relation to $\zeta$ as they do to $\delta$, and in particular $type(x\cup\{\delta\})=type(x\cup \{\zeta\})$.
  
  Thus by mutual homogeneity applied to the formula $``F(\vec{a})<b"$, $\mathcal{M}\models F(x)<\zeta$ and observe that this implies $F(x)<\gamma_n$.
  
  Therefore $\sup(\mathcal{U}\cap \kappa_n)= \gamma_n$ as desired.
\end{proof}

\section{An Indiscernibility Result Applicable to Magidor Forcing}
\label{ms:simultaneous_indiscernibility}

To argue for the mutual stationarity property for Magidor generic sequences, we need an as of yet unpublished simultaneous homogeneity result for multiple normal measures on $\kappa$.
We first argue an ``easy" version for $[\kappa]^{<\omega}$ that pulls at most one ordinal from each measure on $\kappa$, and then generalize to larger sequences.

\begin{definition}
  Let $\seq{X_i\mid 0\leq i<n}$ be sets of ordinals.
  Then we write 
  \[
    \prod^\uparrow_{0\leq i<n} X_i
  \]
  to denote all $n$-length increasing sequences $\beta_0<\dots<\beta_{n-1}$ where $\beta_i\in X_i$.
\end{definition}

\begin{lemma}\thlabel{simult_indisc_easy}
  Suppose $\lambda$ is a regular cardinal below $\kappa$ and $\overrightarrow{U}=\seq{U_\alpha \mid \alpha<\lambda}$ is a sequence of normal measures on $\kappa$ and suppose that $f:[\kappa]^{<\omega}\to\kappa$ is regressive, i.e. for all $x$, $f(x)<\min(x)$.
  
  Then there is an $\overrightarrow{H}=\seq{H_\alpha\mid \alpha<\lambda}$ where each $H_\alpha\in U_\alpha$, 
  such that for every $n$ and every $\alpha_0<\dots<\alpha_{n-1}$,
  \[f\upharpoonright \prod^\uparrow_{0\leq i<n} H_{\alpha_i}\]
  is constant.
\end{lemma}

\begin{proof}
  We proceed by induction on $n$, for all such $f$ simultaneously. 
  The need for all functions simultaneously will be apparent during the inductive step.
  
  If $n=1$, then we need $\tensor*[^1]{\overrightarrow{H}}{}=\seq{\tensor*[^1]{H}{_\alpha}\mid \alpha<\lambda}$ such that for each $\alpha$, $f\upharpoonright \tensor*[^1]{H}{_\alpha}$ is constant. 
  This follows readily by Fodor's theorem, since all the $U_\alpha$'s are $\kappa$-complete and normal.
  
  For the induction hypothesis, suppose that for every $g:[\kappa]^k\to\kappa$ regressive, there is an $\tensor*[^k]{\overrightarrow{H}}{}$ such that for each $\alpha_0<\dots<\alpha_{k-1}$, $g\upharpoonright\prod^\uparrow_{0\leq i<k} \tensor*[^k]{H}{_{\alpha_i}}$ is constant.
  
  For each $\beta_0<\kappa$, let $\tensor*[_{\beta_0}]{f}{}:[\kappa\setminus \{\beta_0\}]^k\to\kappa$, $\tensor*[_{\beta_0}]{f}{}(X)=f(\{\beta_0\}\sqcup X)$.
  
  Since $f$ is regressive, so is $\tensor*[_{\beta_0}]{f}{}$; so by the induction hypothesis, for each $\beta_0$, there is a sequence $\tensor*[^{k}_{\beta_0}]{\overrightarrow{H}}{}$ such that for each $\alpha_1<\alpha_2<\dots<\alpha_k$, $\tensor*[_{\beta_0}]{f}{}\upharpoonright \prod^\uparrow_{1\leq i<k+1} \tensor*[^{k}_{\beta_0}]{H}{_{\alpha_i}}$ is constant, say with value $\gamma_{\alpha_1,\dots,\alpha_k}(\beta_0)$.
  
  Let $g_{\alpha_1,\dots,\alpha_k}:\kappa\to\kappa$, $g_{\alpha_1,\dots,\alpha_k}(\beta)=\gamma_{\alpha_1,\dots,\alpha_k}(\beta)$; observe that $g_{\alpha_1,\dots,\alpha_k}$ is regressive.
  
  Fix an $\alpha<\lambda$; we now wish to define $\tensor*[^{k+1}]{H}{_\alpha}$.
  
  % TODO: change the first \alpha to a superscript to distinguish between \alpha slash \alpha_0 and the tail (\alpha_1 above)
  % for both A_{...}, B_{...}, and \delta_{...}
  By regressiveness of $g_{\alpha_1,\dots,\alpha_k}$, for each $\alpha_1<\dots<\alpha_k$ in $[\lambda\setminus(\alpha+1)]^{<\omega}$, there is an ${A^\alpha_{\alpha_1,\dots,\alpha_k}\in U_{\alpha}}$ such that $g_{\alpha_1,\dots,\alpha_k}$ is constant on $A^\alpha_{\alpha_1,\dots,\alpha_k}$, say with value $\delta^\alpha_{\alpha_1,\dots,\alpha_k}$.
  
  Let \[
    B_{\alpha}=\left(\bigcap_{\alpha<\alpha_1<\dots<\alpha_k<\lambda} A^\alpha_{\alpha_1,\dots,\alpha_k}\right)
  \]
  For each $\beta<\kappa$, let $\tensor*[^{k}_{\beta}]{H}{^{'}_{\alpha}}=\tensor*[^{k}_{\beta}]{H}{_\alpha}\cap B_\alpha$ and define
  \[
    \tensor*[^{k+1}]{H}{_\alpha}=\Delta_{\beta<\kappa}\tensor*[^{k}_{\beta}]{H}{^{'}_{\alpha}}
  \]
  Then $\tensor*[^{k+1}]{H}{_\alpha}$ is in $U_\alpha$, and we now verify that $\tensor*[^{k+1}]{\overrightarrow{H}}{}$ has the appropriate homogeneity.
  Suppose that $\alpha_0<\dots<\alpha_k$.
  We wish to verify that for every $\beta_i\in \tensor*[^{k+1}]{H}{_{\alpha_i}}$, $\beta_i<\beta_{i+1}$, we have $f(\{\beta_0,\dots,\beta_k\})=\delta^{\alpha_0}_{\alpha_1,\dots,\alpha_k}$.
  To see this, observe that for each possible choice of $\beta_0$, $f(\{\beta_0,\dots,\beta_k\})=\tensor*[_{\beta_0}]{f}{}(\{\beta_1,\dots,\beta_k\})$;
  since $\beta_i<\beta_{i+1}$, for $i\geq 1$ we have that $\beta_i\in \tensor*[^{k}_{\beta_0}]{H}{_{\alpha_i}}$.
  Hence $f(\{\beta_0,\dots,\beta_k\})=\gamma_{\alpha_1,\dots,\alpha_k}(\beta_0)$.
  But since $\beta_0\in B_{\alpha_0}$, we have that in fact $\gamma_{\alpha_1,\dots,\alpha_k}(\beta_0)=\delta^{\alpha_0}_{\alpha_1,\dots,\alpha_k}$ as desired.
  
  In the end, our desired $H_\alpha$ is $\bigcap_{n<\omega} \tensor*[^n]{H}{_\alpha}$.
\end{proof}

While \thref{simult_indisc_easy} has some utility, we will need the following more general multi-arity version:

\begin{definition}
  Let $\seq{X_i\mid 0\leq i<n}$ be sets of ordinals and let $k_0,\dots,k_{n-1}$ be natural numbers.
  Then we write 
  \[
    \prod^\uparrow_{0\leq i<n} X_{i}^{k_i}
  \]
  to denote the collection of all $\overrightarrow{\beta_0},\dots,\overrightarrow{\beta_{n-1}}$ which are $k_0,\dots,k_{n-1}$-ary increasing sequences such that $\overrightarrow{\beta_i}\subseteq X_{i}$ and $\max \overrightarrow{\beta_i}<\min \overrightarrow{\beta_{i+1}}$.
\end{definition}

\begin{lemma}\thlabel{simult_indisc_useful}
  Suppose $\lambda$ is a regular cardinal below $\kappa$ and $\overrightarrow{U}=\seq{U_\alpha \mid \alpha<\lambda}$ is a system of normal measures on $\kappa$ and suppose that $f:[\kappa]^{<\omega}\to\kappa$ is regressive.
  
  Then there is an $\overrightarrow{H}=\seq{H_\alpha\mid \alpha<\lambda}$ where each $H_\alpha\in U_\alpha$, 
  such that for every $n$ and for every $k_0,\dots,k_{n-1}$,
  for every $\alpha_0<\dots<\alpha_{n-1}$,
  \[
    f\upharpoonright \prod^\uparrow_{0\leq i<n} H_{\alpha_i}^{k_i}
  \]
  is constant.
\end{lemma}

There was nothing saying that a normal measure \emph{can't} repeat itself in the statement of \thref{simult_indisc_easy}.
That leads to the following short proof, where we repeat the same normal measure $\omega$-many times consecutively:

\begin{proof}
  Apply \thref{simult_indisc_easy} to the sequence
  $\vec{U}'=\seq{U'_{\omega\cdot\alpha+k} \mid k<\omega, \alpha<\lambda}$
  where $U'_{\omega\cdot\alpha+k}=U_\alpha$ for all $k$.
  
  This gives a sequence $\seq{H'_{\omega\cdot\alpha+k}\mid k<\omega,\alpha<\lambda}$ with the homogeneity result for $\vec{U}'$ as in \thref{simult_indisc_easy}.
  To obtain homogeneity as desired for $\vec{U}$,
  we will have that $H_\alpha=\bigcap_{k<\omega} H'_{\omega\cdot\alpha+k}$.
\end{proof}


\section{Mutual Stationarity for Magidor Generics}
\label{ms:magidor_gens}

Argues that Magidor generics satisfy mutual stationarity.

