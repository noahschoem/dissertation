\section{Chapter-specific preliminaries}
\label{destroy_sat:preliminaries}

(TODO: Put the section-specific background here, i.e. stuff that's not fundamental enough to go in the intro but is only applicable to this chapter, and separate out the rest.

Especially some of the notational bs like $Reg_\kappa$, $V^{\mathbb{P}}$.

Also for subject matter: ideals, cplteness, normality, cc, presat, closure, precipitousness, generic ultrapowers, Foreman duality.)

Much of the necessary background can be found in Chapter \ref{forcing}; we cover more chapter-specific background here.

For a cardinal $\kappa$, we will write $Reg_\kappa$ for the set of regular cardinals below $\kappa$, and $cof(\kappa)$ for the proper class of cardinals of cofinality $\kappa$.

If $\mathbb{P}$ is a notion of forcing in $V$, we will variously use $V^\mathbb{P}$ or $V[G]$ to refer to the generic extension of $V$ by $\mathbb{P}$.

The following definition summarizes some forcing properties of posets that will come in handy:

\begin{definition}[Chain condition, presaturation, and closure]
Let $(\mathbb{P},\leq)$ be a poset. We say that:
\begin{enumerate}[label={(\roman*)}]
\item (\cite{baumgartner_taylor_sat_gen_ees_two}, as Theorem 4.2) $\mathbb{P}$ is \emph{$\mu$-presaturated} if for every $\lambda<\mu$ and every family $\seq{A_\alpha \mid \alpha<\lambda}$ of antichains,
there are densely many $p\in\mathbb{P}$ such that for all $\alpha$, $\{q\in A_\alpha \mid p\compat q\}$ has cardinality $<\mu$. Note that $\mu$-cc implies $\mu$-presaturation.

\item $\mathbb{P}$ is \emph{$<\kappa$-closed} if whenever $\tau<\kappa$ and $\seq{p_\alpha \mid \alpha<\tau}$ is a $\leq$-decreasing sequence in $\mathbb{P}$, there is a $p\in\mathbb{P}$ such that $p\leq p_\alpha$ for all $\alpha<\tau$

\item $\mathbb{P}$ is \emph{$<\kappa$-directed closed} ($<\kappa$-dc) if whenever $D\subseteq \mathbb{P}$ is a directed set\footnote{that is, for all $p,q\in D$, there is an $r\in D$ such that $r\leq p,q$} with $|D|<\kappa$, there is a $q\in \mathbb{P}$ such that whenever $p\in D$, $q\leq p$

\item $\mathbb{P}$ is \emph{$\mu$-preserving} (for $\mu$ a $V$-cardinal) if $V^\mathbb{P}\models ``\check{\mu}\text{ is a cardinal}"$
\end{enumerate}
\end{definition}

For arbitrary posets, presaturation can be pushed downwards through a two-step iteration:

\begin{lemma}[Lemma 2.12 of \cite{cox_eskew_kill_sat_save_presat}]\label{pushpresatdown}
If $\mathbb{P} * \dot{\mathbb{Q}}$ is $\kappa$-presaturated then $\mathbb{P}$ is $\kappa$-presaturated and $1_\mathbb{P}\forces \dot{\mathbb{Q}}$ is $\kappa$-presaturated.
\end{lemma}

Whether the converse holds is currently an open problem; this appears as Question 8.6 of \cite{cox_eskew_kill_sat_save_presat}. 

Forcing poset closure, and properness, relate to presaturation; we now summarize what properness is, and how both closure and properness relate to presaturation.

Let $\delta$ be regular uncountable, and let $H\supsetneq\delta$.
Then we write $\mathcal{P}_\delta(H)$ for all subsets of $H$ of size $<\delta$, and $\mathcal{P}^*_\delta(H)$ to denote the set of all $x\in \mathcal{P}_\delta(H)$ such that $x\cap\delta\in\delta$.

\begin{definition}
Let $\mathbb{P}$ be a notion of forcing, $\theta$ sufficiently large so that $\mathbb{P}\in H_\theta$, and $M\prec (H_\theta,\in,\mathbb{P})$. 

We say that $p\in\mathbb{P}$ is an \emph{$(M,\mathbb{P})$-master condition} if for every dense $D\in M$, $D\cap M$ is predense below $p$; equivalently, $p\forces_{\mathbb{P}} M[\dot{G}_{\mathbb{P}}]\cap V=M$. 

Additionally, we say that $p$ is an \emph{$(M,\mathbb{P})$-strong master condition} if for every $p'\leq p$, there is some $p'_M\in M\cap\mathbb{P}$ such that every extension of $p'_M$ in $M\cap\mathbb{P}$ is compatible with $p'$. 
\footnote{It is straightforward to see that strong master conditions are also master conditions.}

Further, $\mathbb{P}$ is \emph{(strongly) proper with respect to $M$} if every $p\in M\cap\mathbb{P}$ has a $q\leq p$ such that $q$ is an $(M,\mathbb{P})$-(strong) master condition.

We say that $\mathbb{P}$ is \emph{(strongly) $\delta$-proper on a stationary set} if there is a stationary subset $S$ of $\mathcal{P}^*_\delta(H_\theta)$ such that for every $M\in S$, $M\prec (H_\theta,\in,\mathbb{P})$ and $\mathbb{P}$ is (strongly) proper with respect to $M$.
\end{definition}

Note that $\{M\in \mathcal{P}^*_\delta(H_\theta)\mid M\prec (H_\theta,\in,\mathbb{P})\}$ is a club subset of $\mathcal{P}^*_\delta(H_\theta)$; so a forcing being $\delta$-proper on a stationary set really only depends on the properness condition.

\begin{fact}
If $\mathbb{P}$ is $\delta$-proper on a stationary set, then $\mathbb{P}$ is $\delta$-presaturated.
\end{fact}

This fact appears as Fact 2.8 of \cite{cox_eskew_kill_sat_save_presat}, with proof; their proof, in turn, generalizes a result of Foreman and Magidor in the case of $\delta=\omega_1$ (namely, Proposition 3.2 of \cite{foreman_magidor_lcs_ctx_ch}).

For the posets we will be working with, we will have a specific stationary subset witnessing $\delta$-properness:

\begin{definition}
For $\delta$ regular and $\theta>>\delta$, we say that $IA_{<\delta}\subseteq \mathcal{P}^*_\delta(H_\theta)$, 
the ``internally approachable sets of length $<\delta$", 
is the collection of all $M\in \mathcal{P}^*_\delta(H_\theta)$, with $|M|=|M\cap \delta|$, that are \emph{internally approachable},
i.e. such that there is a $\zeta<\delta$ and a continuous $\subseteq$-increasing sequence $\seq{N_\alpha\mid \alpha<\zeta}$ whose union is $M$, such that $\vec{N}\upharpoonright \alpha\in M$ for all $\alpha<\zeta$.
\end{definition}

In a sense, internal approachability is preserved by any generic extension:

\begin{fact}\label{principle-A}
Suppose $\mathbb{P}$ is a poset, $M\prec(H_\theta,\in,\mathbb{P})$, $\seq{N_\alpha\mid\alpha<\zeta}$ witnesses that $M\in IA_{<\delta}$, and $G$ is $(V,\mathbb{P})$-generic.
Then in $V[G]$, $\seq{N_\alpha[G]\mid \alpha<\zeta}$ witnesses that $M[G]\in IA_{<\delta}$. (Without loss of generality, we may assume that $\mathbb{P}\in N_0$.)
\end{fact}

It is a standard fact that $IA_{<\delta}$ is stationary.
The following lemma makes clear its utility:

\begin{lemma}\label{lemma-iaposets}
Let $\delta$ be regular and uncountable. Then:
\begin{enumerate}[label={(\roman*)}]
\item\label{lemma-iaposets:cc}
If $\mathbb{P}$ is $\delta$-cc and $M\prec (H_\theta,\in,\mathbb{P})$ is an element of $\mathcal{P}^*_\delta(H_\theta)$ (i.e. if $M\cap\delta\in\delta$), then $1_\mathbb{P}$ is an $(M,\mathbb{P})$-master condition; in particular $\mathbb{P}$ is $\delta$-proper on $\mathcal{P}^*_\delta(H_\theta)$.

\item\label{lemma-iaposets:closed}
If $\mathbb{Q}$ is $<\delta$-closed then $\mathbb{Q}$ is $\delta$-proper on $IA_{<\delta}$.

\item\label{lemma-iaposets:iter}
If $\mathbb{P}$ is $\delta$-proper on $IA_{<\delta}$ and $\forces_\mathbb{P}``\dot{\mathbb{Q}}\text{ is }\delta\text{-cc}$ or $\forces_\mathbb{P}``\dot{\mathbb{Q}}\text{ is }<\delta\text{-closed}$ then $\mathbb{P}*\mathbb{\dot{Q}}$ is $\delta$-proper on $IA_{<\delta}$.
\end{enumerate}
\end{lemma}

This is roughly Fact 2.9 out of \cite{cox_eskew_kill_sat_save_presat}. The following proof is largely reproduced from \cite{cox_eskew_kill_sat_save_presat} as well.

\begin{proof}
For part \ref{lemma-iaposets:cc}, let $A\in M$ be a maximal antichain in $\mathbb{P}$.
Since $|A|<\delta$ and $M\cap\delta\in\delta$, we have that $A\subseteq M$.
Thus $1_\mathbb{P}\forces M[\dot{G}]\cap\check{V}=M$, so $1_\mathbb{P}$ is a master condition for $M$.

Part \ref{lemma-iaposets:closed} is due to Foreman and Magidor in \cite{foreman_magidor_lcs_ctx_ch}. %I can't specifically find it here, but Cox & Eskew say it's in there.

As for part \ref{lemma-iaposets:iter}, let $G$ be $\mathbb{P}$-generic over $V$.
Suppose that $M\prec (H_\theta,\in,\mathbb{P}*\dot{\mathbb{Q}})$ and $M\in IA_{<\delta}$.
By Fact \ref{principle-A}, combined with \ref{lemma-iaposets:cc} and \ref{lemma-iaposets:closed},
$\mathbb{P}$ forces that $\dot{\mathbb{Q}}$ is proper with respect to $M[\dot{G}]$.
Hence $\mathbb{P}*\dot{\mathbb{Q}}$ is proper with respect to $M$.
\end{proof}

Presaturation has a useful corollary:
\begin{fact}
If $\mathbb{P}$ is $\lambda$-presaturated for $\lambda$ regular then 
\[\forces_{\mathbb{P}} cof^V(\geq\lambda)=cof^{V[\dot{G}]}(\geq\lambda)\]
\end{fact}
The above fact has a partial converse. We will not make use of it, but it is another known way to argue that certain iterations of presaturated forcings are presaturated:
\begin{fact}\label{cof->presat}
If $\mathbb{P}$ is $\lambda^{+\omega}$-cc for some regular $\lambda\geq\omega_1$ and
\[\forall n\in\omega \ \forces_{\mathbb{P}} cf^{V[\dot{G}]}\left(\left(\lambda^{+n}\right)^V\right)\geq\lambda\]
then $\mathbb{P}$ is $\lambda$-presaturated.
\end{fact}

This appears as Fact 2.11 in \cite{cox_eskew_kill_sat_save_presat}, which in turn is a generalization of Theorem 4.3 of \cite{baumgartner_taylor_sat_gen_ees_two}.

In line with Section \ref{forcing:ideals_measures}, if $I\in V$ is an ideal on $\kappa$ and $\mathbb{P}$ is a notion of forcing understood from context, then we will write $\overline{I}:=\left\{N\in \mathcal{P}^{V^\mathbb{P}}(\kappa)\mid \exists A\in I \ N\subseteq A\right\}$.

To compute presaturation properties of ideals, these are the two forms of Foreman's Duality Theorem that we will use:

\begin{lemma}\label{foreman_ideals_gen_ees:groundideals}
For a $\kappa$-complete, $\kappa^+$ saturated $I\in V$, $\overline{I}$ is $\kappa^+$-saturated in $V^\mathbb{Q}$ if and only if $\forces_{\mathcal{B}_I} \dot{j}_I(\mathbb{Q})$ is $\kappa^+$-cc.
\end{lemma}
 
This appears as Corollary 7.21 in \cite{foreman_ideals_gen_ees}.

\begin{theorem}
\label{easyduality}
Let $I$ be a $\kappa$-complete normal precipitous ideal in $V$ and $\mathbb{Q}$ be a $\kappa$-cc poset. 
Then $\overline{I}$ is precipitous and there is a canonical isomorphism witnessing that
\[\mathcal{B}\left(\mathbb{Q} * \mathcal{B}_{\overline{I}}\right)\cong \mathcal{B}\left(\mathcal{B}_I * \dot{j}_I\left(\mathbb{Q}\right)\right)\]
where $\mathcal{B}(\mathbb{P})$ refers to the Boolean completion of $\mathbb{P}$.
\end{theorem}

This statement appears in \cite{cox_eskew_kill_sat_save_presat} as Fact 2.24, and is a corollary of Theorem 7.14 of \cite{foreman_ideals_gen_ees}.
